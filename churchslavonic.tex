\newfontfamily\russianfont[Script=Cyrillic,Ligatures=TeX]{Times New Roman}
\newfontfamily\russianfonttt[Script=Cyrillic,Ligatures=TeX]{Courier New}
\newfontfamily\russianfontsf[Script=Cyrillic,Ligatures=TeX]{Arial}
\newfontfamily\churchslavonicfont[Script=Cyrillic,Ligatures=TeX,HyphenChar="005F]{PonomarUnicode.otf}
\usepackage{churchslavonic}
\usepackage{hyperref}
\usepackage{xltxtra}
\usepackage{doc}

\def\pkg#1{\textsf{#1}}
\def\cs#1{\texttt{\textbackslash #1}}
\providecommand{\XeTeX}{X\kern-.125em\lower.5ex\hbox{Ǝ}\kern-.1667em\TeX}
\providecommand{\XeLaTeX}{X\kern-.125em\lower.5ex\hbox{Ǝ}\kern-.125em\LaTeX}
\providecommand{\LuaTeX}{Lua\kern-.125em\TeX}

\begin{document}

\begin{EN}
\title{\pkg{churchslavonic} package --- Church Slavonic Typography in \LaTeX}
\author{Aleksandr Andreev and Mike Kroutikov}
\end{EN}

\begin{RU}
\title{Пакет \pkg{churchslavonic} --- верстка церковнославянских текстов в системе \LaTeX}
\author{Александр Андреев и Михаил Крутиков}
\end{RU}

\date{\today}
\maketitle

\begin{EN}
\begin{abstract}
Package \pkg{churchslavonic} provides fonts, hyphenation patterns and supporting macros to typeset
Church Slavonic texts.
\end{abstract}
\end{EN}

\begin{RU}
\begin{abstract}
Пакет \pkg{churchslavonic} позволяет верстать церковнославянские документы. В пакет включены шаблоны переносов,
шрифты и набор необходимых макрокоманд.
\end{abstract}
\end{RU}

\tableofcontents

\begin{EN}
\section{Introduction}
Church Slavonic (also called Church Slavic, Old Church Slavonic
or Old Slavonic; ISO 639-2 code \texttt{cu}) is a literary language used by
the Slavic peoples; presently it is used as a liturgical language by the
Russian Orthodox Church, other local Orthodox Churches, as well
as various Byzantine-Rite Catholic and Old Ritualist communities.
The package \texttt{churchslavonic} provides fonts, hyphenation
patterns and supporting macros to typeset
Church Slavonic texts in \TeX{}.

The package is designed to support Unicode text encoded encoded in UTF-8.
Texts encoded in legacy codepages (such as HIP and UCS) may be
converted to Unicode using a separate bundle of utilities.
See the \href{http://sci.ponomar.net/}
{Slavonic Computing Initiative website} for more information.
To use the tools in this package, you will need a Unicode-aware \TeX{}
engine such as \XeTeX{} or \LuaTeX{}.
\end{EN}

\begin{RU}
\section{Введение}
Церковнославянский язык (ISO 639-2 код \texttt{cu}) --
древний литературный язык славянских народов,
который теперь используется в основном как богослужебный язык
в Русской Православной Церкви, других поместных православных
церквях, а также в грекокатолических и старообрядческих общинах.
Пакет \pkg{churchslavonic} позволяет верстать церковнославянские тексты
в системе \TeX{}. В пакет включены шаблоны переносов,
шрифты и набор необходимых макрокоманд.

Данный пакет поддерживает только набор в стандарте Юникод.
Тексты, набранные в устаревших кодировках HIP и UCS,
можно конвертировать в Юникод (точнее -- в UTF-8) с помощью
отдельных утилит, которые можно найти на сайте
\href{http://sci.ponomar.net/}
{Инициативной группы славянской информатики}.
Для того, чтобы использовать возможности этого пакета Вам
понадобится система верстки \TeX{}, поддерживающая Юникод,
например, \XeTeX{} или \LuaTeX{}.
\end{RU}

\begin{EN}
\section{Numbers}

Church Slavonic numbering system is heavily based on the Old Greek one and uses 
letters as digits.
For more information on the matter, see the appropriate section in \ref{UN41}.
\end{EN}

\begin{RU}
\section{Числа}
Способ записи чисел в церковнославянском языке основывается на древнегреческом и в
качестве цифр использует буквы. За подробностями отсылаем интересующегося читателя к
соответствующей главе \ref{UN41}.
\end{RU}

\begin{center}
\begin{churchslavonic}
\begin{tabular}{| r | l |}
\hline
\cuNum{1} & \textenglish{1} \\
\cuNum{12} & \textenglish{12} \\
\cuNum{123} & \textenglish{123} \\
\cuNum{1234} & \textenglish{1234} \\
\cuNum{10345} & \textenglish{10345} \\
\cuNum{12345} & \textenglish{12345} \\
\cuNum{123456} & \textenglish{123456} \\
\cuNum{800456} & \textenglish{800456} \\
\cuNum{1234567} & \textenglish{1234567} \\
\cuNum{1500567} & \textenglish{1500567} \\
\cuNum{12345678} & \textenglish{12345678} \\
\cuNum{123456789} & \textenglish{123456789} \\
\hline
\end{tabular}
\end{churchslavonic}
\end{center}

\begin{EN}
\subsection{\cs{cuNum} command}
Use this command to typeset a Church Slavonic number.
The command takes a single argument that should expand to a number (register name works too).

For example, the left column of the third row in the table above was typset as
\begin{verbatim}
\cuNum{123}
\end{verbatim}
\end{EN}

\begin{RU}
\subsection{Команда \cs{cuNum}}
Команда печатает число в церковнослявянской нотации.
Она принимает единственный аргумент. Аргументом может быть текст, командная последовательность, или имя 
целочисленного регистра. Единственное условие --- аргумент должен раскрыться в число.

Например, содержимое левой колонки третьего ряда в таблице было получено с помощью
следующей команды:
\begin{verbatim}
\cuNum{123}
\end{verbatim}
\end{RU}

\begin{EN}
\section{Dates}
\subsection{\cs{cuDate}}
Command formats the date (according to the current format). Argument is a triplet of numbers \texttt{YYYY-MM-DD} specifying
the date. Output will be something like this: \textchurchslavonic{\cuDate{2016-04-22}}.
\end{EN}

\begin{RU}
\subsection{\cs{cuDate}}
Команда форматирует дату (в соответствии с текущим форматом). Аргумент должен иметь вид \texttt{YYYY-MM-DD}. Результат
может выглядеть примерно так: \textchurchslavonic{\cuDate{2016-04-22}}.
\end{RU}

\begin{EN}
\subsection{\cs{cuDefineDateFormat}}
Command allows one to define date format. It does not change how \cs{cuDate} formats its output (for that, use \cs{cuUseDateFormat}).

Example:
\end{EN}

\begin{RU}
\subsection{\cs{cuDefineDateFormat}}
Команда определяет формат даты. Она никак не влияет на то как \cs{cuDate} форматирует свой вывод (для этого используется \cs{cuUseDateFormat}).

Пример:
\end{RU}

\begin{churchslavonic}
\noindent
\verb+\cuDefineDateFormat{long}{%+\\
\verb+  \cuDayName{\cuDOW},+\\
\verb+  \cuNum{\cuDAY}+_гѡ\verb+~%+\\
\verb+  \cuMonthName{\cuMONTH}~%+\\
\verb+  +лѣ́та ѿ сотворе́нїѧ мі́ра\verb+~%+\\
\verb+  \cuNum{\cuYEARAM}%+\\
\verb+}+\\
\end{churchslavonic}
%
\cuDefineDateFormat{long}{%
  \cuDayName{\cuDOW},
  \cuNum{\cuDAY}_гѡ~%
  \cuMonthName{\cuMONTH}~%
  лѣ́та ѿ сотворе́нїѧ мі́ра~\cuNum{\cuYEARAM}%
}
%
\begin{EN}
defines a format with name \texttt{long}. If we use this format to print the same date as before, we will get:
\textchurchslavonic{\cuUseDateFormat{long}\cuDate{2016-04-22}}.
\end{EN}

\begin{RU}
определяет новый формат с именем \texttt{long}. Если мы напечатаем дату этим форматом, то получим:
\textchurchslavonic{\cuUseDateFormat{long}\cuDate{2016-04-22}}.
\end{RU}

\begin{EN}
Following symbolic names can be used when formatting the date:
\begin{itemize}
\item \cs{cuYEAR} --- year part of date (number, like \texttt{2016})
\item \cs{cuYEARAM} --- year since world creation, aka Anno Mundi (number, like \texttt{7525})
\item \cs{cuMONTH} --- month part of date (number from 1 to 12)
\item \cs{cuDAY} --- day of the month
\item \cs{cuDOW} --- day of the week (number from 0 to 6, where 0 means "Sunday")
\item \cs{cuINDICTION} --- indiction year (number from 1 to 15)
\end{itemize}
\end{EN}

\begin{RU}
При определении формата можно пользоваться следующими символическими именами:
\begin{itemize}
\item \cs{cuYEAR} --- год (число, например \texttt{2016})
\item \cs{cuYEARAM} --- год от сотворения мира, Anno Mundi (число, например \texttt{7525})
\item \cs{cuMONTH} --- месяц (число от 1 до 12)
\item \cs{cuDAY} --- день месяца
\item \cs{cuDOW} --- день недели (число от 0 to 6, где 0 означает "воскресение")
\item \cs{cuINDICTION} --- индикт (число от 1 до 15)
\end{itemize}
\end{RU}

\begin{EN}
\subsection{\cs{cuUseDateFormat}}
\subsection{\cs{cuJulianDate}}
\subsection{\cs{cuIndiction}}
\subsection{\cs{cuMonthName}}
\subsection{\cs{cuDayName}}
\subsection{\cs{cuDayNameAccusative}}
\subsection{\cs{cuToday}}
\end{EN}

\begin{RU}
\section{Даты}
\subsection{\cs{cuDate}}
\subsection{\cs{cuDefineDateFormat}}
\subsection{\cs{cuUseDateFormat}}
\subsection{\cs{cuJulianDate}}
\subsection{\cs{cuIndiction}}
\subsection{\cs{cuMonthName}}
\subsection{\cs{cuDayName}}
\subsection{\cs{cuDayNameAccusative}}
\subsection{\cs{cuToday}}
\end{RU}

\begin{center}
\begin{churchslavonic}
\begin{tabular}[]{ | r | l | }
\hline
\cuDate{2016-4-21} & \verb+\cuDate{2016-4-21}+ \\
\cuJulianDate{2016-4-21} & \verb+\cuJulianDate{2016-4-21}+ \\
\cuDate{\cuToday} & \verb+\cuDate{\cuToday}+ \\
\hline
\end{tabular}
\end{churchslavonic}
\end{center}

\begin{EN}
\section{Utilities}
\end{EN}

\begin{RU}
\section{Утилиты}
\end{RU}

\begin{EN}
\section{Kinovar}
\subsection{\cs{cuKinovar}}
\subsection{\cs{cuKinovarColor}}
\end{EN}

\begin{RU}
\section{Киноварь}
\subsection{\cs{cuKinovar}}
\subsection{\cs{cuKinovarColor}}
\end{RU}

\end{document}
