\usepackage{churchslavonic}
\usepackage{hyperref}

\begin{document}

\begin{EN}
\title{\textsf{churchslavonic} package --- Church Slavonic Typography in \LaTeX}
\author{Aleksandr Andreev and Mike Kroutikov}
\end{EN}

\begin{RU}
\title{Пакет \textsf{churchslavonic} --- верстка церковнославянских текстов в системе \LaTeX}
\author{Александр Андреев и Михаил Крутиков}
\end{RU}

\date{\today}
\maketitle

\begin{EN}
\begin{abstract}
Package \textsf{churchslavonic} provides fonts, hyphenation patterns and supporting macros to typeset
Church Slavonic texts.
\end{abstract}
\end{EN}

\begin{RU}
\begin{abstract}
Пакет \textsf{churchslavonic} позволяет верстать церковнославянские документы. В пакет включены шаблоны переносов,
шрифты и набор необходимых макрокоманд.
\end{abstract}
\end{RU}

\tableofcontents

\begin{EN}
\section{Introduction}
Church Slavonic is primarily a liturgical language....

Only Unicode input encoding is supported.

Credits to the authors of HIPTEX, CLSTEX, Hirmologion fonts, etc., etc. 
\end{EN}

\begin{RU}
\section{Введение}
Церковнославянский язык --- это богослужебный язык Православной церкви.

Данный пакет поддерживает только набор в стандарте Unicode. Тексты набранные в кодировках HIP и UCS
можно конвертировать в Unicode (точнее --- в UTF8) с помощью отдельных утилит (не включенных в данный пакет).

Выражаем благодарность создателям пакетов HIPTEX, CSLTEX, шрифтов семейства Ирмологион, и всем и вся.
\end{RU}

\begin{EN}
\section{Numbers}

Church Slavonic numbering system is heavily based on the Old Greek one and uses 
letters as digits.
For more information on the matter, see the appropriate section in \ref{UN41}.
\end{EN}

\begin{RU}
\section{Числа}
Способ записи чисел в церковнославянском языке основывается на древнегреческом и в
качестве цифр использует буквы. За подробностями отсылаем интересующегося читателя к
соответствующей главе \ref{UN41}.
\end{RU}

\begin{center}
\begin{churchslavonic}
\begin{tabular}{| r | l |}
\hline
\cuNum{1} & \textenglish{1} \\
\cuNum{12} & \textenglish{12} \\
\cuNum{123} & \textenglish{123} \\
\cuNum{1234} & \textenglish{1234} \\
\cuNum{10345} & \textenglish{10345} \\
\cuNum{12345} & \textenglish{12345} \\
\cuNum{123456} & \textenglish{123456} \\
\cuNum{800456} & \textenglish{800456} \\
\cuNum{1234567} & \textenglish{1234567} \\
\cuNum{1500567} & \textenglish{1500567} \\
\cuNum{12345678} & \textenglish{12345678} \\
\cuNum{123456789} & \textenglish{123456789} \\
\hline
\end{tabular}
\end{churchslavonic}
\end{center}

\begin{EN}
\subsection{\texttt{\textbackslash cuNum} command}
Use this command to typeset a Church Slavonic number.
The command takes a single argument that should expand to a number (register name works too).

For example, the left column of the third row in the table above was typset as
\begin{verbatim}
\cuNum{123}
\end{verbatim}
\end{EN}

\begin{RU}
\subsection{Команда \texttt{\textbackslash cuNum}}
Команда печатает число в церковнослявянской нотации.
Она принимает единственный аргумент. Аргументом может быть текст, командная последовательность, или имя 
целочисленного регистра. Единственное условие --- аргумент должен раскрыться в число.

Например, содержимое левой колонки третьего ряда в таблице было получено с помощью
следующей команды:
\begin{verbatim}
\cuNum{123}
\end{verbatim}
\end{RU}


\end{document}
