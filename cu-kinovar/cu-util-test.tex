\documentclass[12pt]{article}

\usepackage{cu-util}

\begin{document}
\makeatletter

\def\cu@eatall #1\relax{}

\cu@ifnextdigraph{\fail}{\cu@eatall} a\relax
\cu@ifnextdigraph{\fail}{\cu@eatall}a\relax
\cu@ifnextdigraph{\fail}{\cu@eatall}о\relax
\cu@ifnextdigraph{\fail}{\cu@eatall}\bf Zulu\relax
\cu@ifnextdigraph{\cu@eatall}{\fail}оу\relax

%
\cu@ifnextletter{\cu@eatall}{\fail}A\relax
\cu@ifnextletter{\cu@eatall}{\fail}Отец\relax
\cu@ifnextletter{\fail}{\cu@eatall} \relax
\cu@ifnextletter{\fail}{\cu@eatall}\bf\relax
\cu@ifnextletter{\fail}{\cu@eatall}'\relax

% for simplicity, lets declare digits to be "accents"
\cu@declare@accent{o1}%
\cu@declare@accent{o2}%
\cu@declare@accent{o3}%

%\cu@ifnextaccent{\cu@eatall}{\fail}1\relax
%\cu@ifnextaccent{\fail}{\cu@eatall}A\relax
%\cu@ifnextaccent{\fail}{\cu@eatall} Zulu\relax
%\cu@ifnextaccent{\fail}{\cu@eatall}\bf Zulu\relax

% test cu@ifnextbgroup - that checks if next token is a bgroup or not
\cu@ifnextbgroup{\fail}{\cu@eatall}\bf\relax
\cu@ifnextbgroup{\fail}{\cu@eatall} {1}\relax
\cu@ifnextbgroup{\cu@eatall}{\fail}{1}\relax

% test that letter tokenizer correctly merges all digraphs and accents
% and is not confused by spaces or controls
\cu@test@tokenizeletter{A123BCD}{A123-BCD}
\cu@test@tokenizeletter{A123BCD}{A123-BCD}
\cu@test@tokenizeletter{A12{}3456}{A12-3456}
\cu@test@tokenizeletter{A12 3456}{A12- 3456}
\cu@test@tokenizeletter{A12\relax 3456}{A12-\relax 3456}
\cu@test@tokenizeletter{A123 456}{A123- 456}
\cu@test@tokenizeletter{AB123456}{A-B123456}

%
% now some real tests 
%
\cu@test@tokenizeletter{А҆́велевѣ}{А҆́-велевѣ}
\cu@test@tokenizeletter{Оу́вы}{Оу́-вы}
\cu@test@tokenizeletter{оу́вы}{оу́-вы}
\cu@test@tokenizeletter{{о}у́вы}{о-у́вы}


\makeatother
\end{document}